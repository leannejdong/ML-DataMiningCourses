%\documentclass[12pt, answers]{exam}
\documentclass[12pt, answers]{exam}
%\setlength{\textheight}{5in}
%\usepackage[textwidth=9in,centering]{geometry}
\usepackage[a4paper, margin=.9in]{geometry}
\usepackage{econometrics}
\usepackage{lmodern}
\usepackage{amssymb,amsmath,amsfonts,amsthm,enumerate,enumitem}
\usepackage{setspace, graphicx, multirow, dcolumn, caption}
\usepackage{bm}


% \usepackage{etoolbox}
% \AtEndEnvironment{question}{\bigskip\bigskip}

% \usepackage{etoolbox}
% \makeatletter
%  \patchcmd\@setheadheight{\endgroup}{\global\vsize=\textheight\endgroup}{}{}
% \makeatother


% Notation
\def\eps{\varepsilon}
\def\var{\text{Var}}
\def\cov{\text{Cov}}
\def\corr{\text{Corr}}
\def\Ybar{\overline{Y}}
\def\Xbar{\overline{X}}
\def\ybar{\overline{y}}
\def\xbar{\overline{x}}
\def\Yhat{\widehat{Y}}
\def\yhat{\widehat{y}}
\def\fhat{\widehat{f}}
\def\bhat{\widehat{\beta}}
\def\betahat{\widehat{\vbeta}}
\def\MSE{\text{MSE}}
\def\SE{\text{SE}}
\def\train{\mathcal{T}}
\def\data{\mathcal{D}}
\def\model{\mathcal{M}}
\def\hvtheta{\widehat{\vtheta}}
\def\({\left(}
\def\){\right)}


%\setlength{\parindent}{0pt}
\onehalfspace

%\renewcommand{\labelitemi}{1.}

%\bracketedpoints
\pointpoints{mark}{marks}
% \qformat{\textbf{Problem \thequestion: \\ \hspace{10pt} \\}}


\qformat{
    \textbf{Question \thequestion}
    \hfill
    \vrule depth 20pt width 0pt % Large depth to make space
}

\begin{document}

\begin{center}
{\Large \textbf{QBUS6810 Statistical Learning and Data Mining}}\\ \large{Tutorial 2 (Written Exercise)}
\end{center}


%\begin{questions}

%\bigskip

%\question
\noindent Decompose the Expected Prediction Error into three parts: Irreducible error, squared Bias, and Variance (slides 33-35 of Lecture 1).

\begin{solution}
We treat $\vx_0$ and $f$ as fixed, i.e. non-random.  Note that $Y_0$ is not in the original sample, and so $\fhat$ is independent of both $Y_0$ and $\eps_0$.  We have:
\begin{align*}
\text{EPE}=&\,E\left[\(Y_0-\fhat(\vx_0)\)^2\right] = E\left[\(f(\vx_0)+\eps_0-\fhat(\vx_0)\)^2\right] = E\left[\(\eps_0+\(f(\vx_0)-\fhat(\vx_0)\)\)^2\right]\\
=&\,E\left[\eps_0^2\right]+2E\left[\eps_0\(f(\vx_0)-\fhat(\vx_0)\)\right]+E\left[\(f(\vx_0)-\fhat(\vx_0)\)^2\right].
\end{align*}
Note that $2E\left[\eps_0\(f(\vx_0)-\fhat(\vx_0)\)\right]=0$, because $\eps_0$ is independent from~$\(f(\vx_0)-\fhat(\vx_0)\)$, and $E\left[\eps_0\right]=0$, which implies $E\left[\eps_0\(f(\vx_0)-\fhat(\vx_0)\)\right]=E\left[\eps_0\right]E\left[f(\vx_0)-\fhat(\vx_0)\right]=0$.  Also note that, by definition, $\var\left(\eps_0\right)=E\left[\eps^2_0\right]-\Big(E\left[\eps_0\right]\Big)^2$.  Thus, $E\left[\eps_0^2\right]=\var\left(\eps_0\right)=\sigma^2$.  Consequently,
\begin{align*}
\text{EPE}=&\sigma^2+E\left[\(f(\vx_0)-\fhat(\vx_0)\)^2\right]=\text{Irreducible error }+\text{ Reducible error}.
\end{align*}
Now we focus on the Reducible error:
\begin{align*}
&E\left[\(f(\vx_0)-\fhat(\vx_0)\)^2\right]=E\left[\Big(f(\vx_0)-E\big[\fhat(\vx_0)\big]+E\big[\fhat(\vx_0)\big]-\fhat(\vx_0)\Big)^2\right]\\
&=\Big(f(\vx_0)-E\big[\fhat(\vx_0)\big]\Big)^2+2E\left[\Big(f(\vx_0)-E\big[\fhat(\vx_0)\big]\Big)\Big(E\big[\fhat(\vx_0)\big]-\fhat(\vx_0)\Big)\right]\\
&+E\left[\Big(E\big[\fhat(\vx_0)\big]-\fhat(\vx_0)\Big)^2\right].
\end{align*}
The middle term is zero again.  To see this note that the only random component in this term is $\fhat(\vx_0)$, and $E\Big[E\big[\fhat(\vx_0)\big]-\fhat(\vx_0)\Big]=0$.  Hence, the Reducible error is:
\begin{align*}
&\Big(f(\vx_0)-E\big[\fhat(\vx_0)\big]\Big)^2+E\left[\Big(E\big[\fhat(\vx_0)\big]-\fhat(\vx_0)\Big)^2\right]\\
\;\;&=\Big(E\big[\fhat(\vx_0)\big]-f(\vx_0)\Big)^2+E\left[\Big(\fhat(\vx_0)-E\big[\fhat(\vx_0)\big]\Big)^2\right]\\
\;\;&=\text{Bias}^2\Big(\fhat(\vx_0)\Big)+\var\Big(\fhat(\vx_0)\Big)
\end{align*}

Putting it all together,
\begin{align*}
\text{EPE}=\sigma^2+\text{Bias}^2\Big(\fhat(\vx_0)\Big)+\var\Big(\fhat(\vx_0)\Big).
\end{align*}


\end{solution}


%\end{questions}

\end{document} 